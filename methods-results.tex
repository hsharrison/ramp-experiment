\section{Method}
\subsection{Participants}
Ten University of Connecticut undergraduate students participated, in partial fulfillment of a course requirement. Six participants were male and four female.

\subsection{Materials}
A wooden dowel ###~cm in length and ##~cm in diamter was used for dynamic-touch trials. A felt blindfold was used on dynamic-touch trials to eliminate visual information, and on visual trials while the experimenters adjusted the slope of the surface. Participants wore circumaural headphones playing white noise to eliminate auditory information while the experimenters adjusted the slope. Beeping noises were also played through the headphones to signal the start of each trial.  Participants indicated their responses by pressing a button on a Logitech wireless presentation remote. A PC running custom software written in Python controlled the audio and recorded participants' responses and reaction times.

\subsection{Apparatus}
The apparatus consisted of a solid wooden ##~cm x ##~cm platform leaned on a heavy metal block. The slope of the platform was adjusted by sliding the block along the floor. For seven angles of inclination---$12^\circ$, $17^\circ$, $22^\circ$, $27^\circ$, $33^\circ$, $39^\circ$, and $45^\circ$, the corresponding position of the block was determined and marked on the floor so it could be quickly recreated. The apparatus was strong and stable enough to support a person's weight. A rubber mat was attached to the platform in order to increase friction.

\subsection{Design and procedure}
The participant's task in this experiment was to determine whether a surface would support stable upright posture, defined as standing with feet flat and parallel without bending at the hip. With these restrictions, the slope can be accomodated primarily by bending at the ankle \cite{riccio1988}.

The participant perceived the slope either visually or haptically, by exploring the surface with a dowel. In both types of trials, participants stood with their heels 1~m away from the slope. They wore a blindfold and listened to white noise while the experimenters adjusted the slope of the surface. For haptic trials, the participant was instructed to hold the dowel touching the floor and the side of their foot during this time. An experimenter pressed a button on the PC to indicate that the slope was ready, pausing the white noise. After a random delay of 0.5~s-1.5~s, the participant heard a beep indicating the start of the trial. For visual trials, the participant lifted the blindfold at the sound of the beep; for haptic trials, the participant lifted the dowel and began exploring the slope.

The participant indicated their response by pressing one of two buttons on the presentation remote, having previously been instructed which button indicated that the ramp could support upright posture and which that it could not. The participant then indicated their confidence in their judgment by speaking aloud a number from one to seven, where one indicated they were basically guessing and seven indicated they were very confident. An experimenter entered this number into the PC, at which point the white noise was resumed and the participant returned the blindfold to their eyes or the dowel to its initial position. The PC then displayed the angle of inclination for the next trial.

Reaction time was recorded beginning with the beep and ending when a button on the presentation remote was pressed. Participants were not asked to respond quickly; the instructions did not address the speed of the response in any way.

Seven angles of inclination were crossed with two modes of perception for a total of 14 experimental conditions. Each condition was repeated three times for a total of 42 trials per participant. Haptic and visual trials were conducted in separate blocks. Half the participants began with the visual trials and half began with the haptic trials. Angle of inclination was randomized within each block.

At the end of each session, the participant's maximum afforded angle of inclination was estimated. Beginning with the smallest angle, the participant attempted to stand on the platform. If the participant was able to achieve stable upright posture, the angle of inclination was increased. If the participant was not able to achieve stable upright posture within two attempts, the previous angle was recorded as the participant's maximum stable angle of inclination.